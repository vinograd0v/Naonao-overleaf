%!TEX root = ../main.tex

\section{Problema 4: Silueta de la Mano}
Para dibujar la silueta de su mano, siga los siguientes pasos:

\begin{itemize}
    \item Preparamos una tabla de abcisas y ordenadas usando los siguientes comandos de MATLAB:
    \begin{verbatim}
    figure('position',get(0,'screensize'))
    axes('position',[0 0 1 1])
    [x,y] = ginput;
    \end{verbatim}
    \item Dibuje su mano en un papel y póngalo sobre la pantalla del computador. Use el ratón para seleccionar alrededor de 37 puntos que delineen su mano (como se muestra en la figura). Termine la instrucción \texttt{ginput} oprimiendo enter.
    \item Grafique los puntos $(x, y)$ obtenidos y la mano correspondiente mediante el comando \texttt{plot} de MATLAB.
    \item Implemente el método de splines cúbicos.
    \item Interpole por separado los puntos $(i, x_i)$ e $(i, y_i)$ mediante splines cúbicos usando su programa.
    \item Grafique la curva parametrizada que se obtiene.
    \item Estime el área de su mano usando la fórmula del área de Gauss:
    \[
    A = \frac{1}{2} \left| \sum_{i=1}^{n-1} x_i y_{i+1} + x_n y_1 - \sum_{i=1}^{n-1} x_{i+1} y_i - x_1 y_n \right|.
    \]
\end{itemize}
