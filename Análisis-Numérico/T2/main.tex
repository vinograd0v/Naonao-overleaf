\documentclass[12pt, a4paper]{article}%Tipo de documento y tamaño de letra%
\usepackage[spanish]{babel}
\usepackage{color,graphicx}%Para poder insertar graficas%
\usepackage[ruled,vlined]{algorithm2e}
\usepackage{amsmath,amssymb}%Insertar unos símbolos matemáticos especiales%
%%%%%%%%%%%%%%%% LA INTEGRAL COMEMIERDA ESA%%

\def\Xint#1{\mathchoice
{\XXint\displaystyle\textstyle{#1}}%
{\XXint\textstyle\scriptstyle{#1}}%
{\XXint\scriptstyle\scriptscriptstyle{#1}}%
{\XXint\scriptscriptstyle\scriptscriptstyle{#1}}%
\!\int}
\def\XXint#1#2#3{{\setbox0=\hbox{$#1{#2#3}{\int}$ }
\vcenter{\hbox{$#2#3$ }}\kern-.6\wd0}}
\def\ddashint{\Xint=}
\def\dashint{\Xint-}

%%%%%%%%%%%%%%%%%%%%%%%%%%%%%%%%%%%%%%%%%%%%
%para poner código xd
\usepackage{listings}
\lstset{
    language=Matlab,
    basicstyle=\ttfamily\small,
    numbers=left,
    stepnumber=1,
    numbersep=10pt,
    frame=single,
    breaklines=true,
    showstringspaces=false,
    keywordstyle=\color{blue!75!black},
    commentstyle=\color{green!50!black},
    stringstyle=\color{red},
}
\usepackage{silence}
\WarningFilter{latexfont}{Font shape `T1/ntxtlf/m/up' undefined}
\WarningFilter{latexfont}{Some}
\usepackage{setspace}
\usepackage[most]{tcolorbox}
\usepackage[T1]{fontenc}
\usepackage{newtxtext,euler}
\let\oldstylenums\oldstyle
\usepackage{array}
\usepackage{layout}
\usepackage{float}
\usepackage{xcolor}
\usepackage{tikz}
\usepackage{amsthm}
\usepackage{enumerate}
\usepackage{picinpar}
\pagestyle{empty}
\input{boxes}

\usepackage{geometry}
 \geometry{
 a4paper,
 total={170mm,245mm},
 left=25mm,
 top=30mm,
 }

\usepackage{bbm}

\begin{document}

\setlength{\parindent}{0cm}
\hoffset-0.46cm
\voffset-1.46cm

\begin{window}[0,l,{\includegraphics[scale=0.31]{logo.eps}},]
\large\scshape  \hspace{0.4cm}\textsf{Universidad Nacional de Colombia} \\
\textcolor{white}{\tiny.}  \large \hspace{1.5cm} \textsf{Facultad de Ciencias} \\
\textcolor{white}{\tiny.}   \normalsize\hspace{2.2cm}\textsf{Análisis Numérico I}\\
 

\end{window}

\vspace{0.2cm}
\small
\textsf{Mateo Andrés Manosalva Amaris\\
Edgar Santiago Ochoa Quiroga\\
Sergio Alejandro Bello Torres} 
\normalsize
\dotfill
\vspace{0.7cm}

\section*{\textcolor{white}{Ejercicios}}
\vspace*{-2cm} %esto es porque me raya que la sección salga y tampoco quiero que quede espacio xd

%La madre para el que mueva algo en el main.tex
%!TEX root = ../main.tex

\subsection*{Ejercicio 1.}
Sea $A \in \mathbb{R}^{m \times n}$. Entonces se satisface:

\begin{enumerate}
    \item[(a)] $\|A\|_2 = \|A^T\|_2 \leq \|A\|_F = \|A^T\|_F$
    \begin{proof}
    Para mostrar que $\|A\|_2 = \|A^T\|_2$ veremos que todo valor propio no nulo de $A^TA$ lo es también de $AA^T$: Sea $v$ un vector propio de $A^TA$ y $\lambda$ el valor propio de $A^TA$ asociado a $v$, entonces
    \begin{align*}
    AA^T(Av) &= A(A^TAv) \\ 
        & = A(\lambda v) \\ 
        & = \lambda (Av) 
    \end{align*}
    De esto se sigue que $\lambda$ es un valor propio de $AA^T$ asociado al vector propio $Av$. De manera análoga, se concluye que si $w$ es un vector propio de $AA^T$, con $\gamma$ el valor propio de $AA^T$ asociado a $w$, entonces $\gamma$ es un valor propio de $AA^T$ asociado al vector propio $A^Tw$.

    Para ver que $\|A\|_F = \|A^T\|_F$ basta notar que
    \[
    \|A\|_F^2=\sum_{i=1}^{m} \sum_{j=1}^{n} |a_{ij}|^2 = \sum_{j=1}^{n} \sum_{i=1}^m |a_{ij}|^2=\sum_{j=1}^{n} \sum_{i=1}^m |a_{ji}|^2=\|A^T\|_F^2
    \]

    Por último, note que las componentes de la diagonal de $A^TA$ son de la forma $u_{jj}=\sum_{i=1}^m a_{ij}^2$, por lo tanto la traza de $A^TA$ es
    \[
    \sum_{j=1}^n\sum_{i=1}^{m}|a_{ij}|^2=\|A\|_F^2
    \]
    y ya que la traza de una matriz es igual a la suma de sus valores propios, y como $A^TA$ es semidefinida positiva, obtenemos que: 
    \begin{align*}
    \lambda_{max}(A^TA)&\leq \text{tr}(A^TA) \\
    \sqrt{\lambda_{max}(A^TA)} & \leq \sqrt{\text{tr}(A^TA)}\\ 
    \|A\|_2\ & \leq \|A\|_F
    \end{align*}
    \end{proof}
    \item[(b)] $\|A\|_\infty \leq \sqrt{n} \|A\|_2$
    \begin{proof}
    Primero note que $\|Ae_j\|_2^2=\sum_{i=1}^m |a_{ij}|^2\leq \|A\|_2^2$, por lo tanto, $\sum_{j=1}^n \|Ae_j\|_2^2\leq n\|A\|_2^2$. Ahora, usando la desigualdad de Cauchy-Schwartz obtenemos el siguiente resultado:
    \begin{align*}
    \|A\|_\infty^{2} & =\left(\max_{1\leq i \leq m}{\sum_{j=1}^n|a_{ij}|}\right)^2\\ 
        &\leq \max_{1\leq i \leq m}{\sum_{j=1}^n|a_{ij}|^2}\\   
        &\leq \sum_{i=1}^m\sum_{j=1}^{n}|a_{ij}|^2\\ 
        &= \sum_{j=1}^n\sum_{i=1}^{m}|a_{ij}|^2\\
        &= \sum_{j=1}^n \|Ae_j\|_2^2\\
        &\leq n\|A\|_2^2
    \end{align*}
    Tomando raíz cuadrada llegamos a que $\|A\|_\infty \leq \sqrt{n}\|A\|_2$.
    \end{proof}
    \item[(c)] $\|A\|_2 \leq \sqrt{m} \|A\|_\infty$
    \begin{proof}
    Note que
    \begin{align*}
    \|Ax\|_2^2 &= \sum_{j=1}^m\left|\sum_{i=1}^n a_{ij}x_{j}\right|^2\\ 
    &\leq \sum_{j=1}^m\left(\left|\max_{1\leq i \leq m}{\sum_{j=1}^n a_{ij}x_{j}}\right|^2\right)\\ 
    &= m\left|\max_{1\leq i \leq m}{\sum_{j=1}^n a_{ij}x_{j}}\right|^2\\ 
    &= m\|Ax\|_\infty^2
    \end{align*}
    Tomando raíz cuadrada se obtiene que $\|Ax\|_2\leq \sqrt{m}\|Ax\|_\infty$ y por lo tanto $\|A\|_2 \leq\sqrt{m} \|A\|_\infty$
    \end{proof}
    \item[(d)] $\|A\|_2 \leq \sqrt{\|A\|_1 \|A\|_\infty}.$
    \begin{proof}
    Tenemos que $\|Ax\|_\infty = \max_{1 \leq i \leq n}{\left|\sum_{j=1}^{n}a_{ij}x_j\right|}$ y también que $\|Ax\|_1 = \sum_{i=1}^m\left|\sum_{j=1}^{n}a_{ij}x_{j}\right|$. De esta manera
    \begin{align*}
    \|Ax\|_2^2 & = \sum_{i=1}^m\left|\sum_{j=1}^{n}a_{ij}x_{j}\right|^2 \\ 
    & = \sum_{i=1}^m\left(\left|\sum_{j=1}^{n}a_{ij}x_{j}\right|\left|\sum_{j=1}^{n}a_{ij}x_{j}\right|\right) \\ 
    &\leq \left(\sum_{i=1}^m\left|\sum_{j=1}^{n}a_{ij}x_{j}\right|\right)\max_{1\leq i \leq m}{\left|\sum_{j=1}^{n}a_{ij}x_{j}\right|} \\ 
    & = \|Ax\|_1 \|Ax\|_\infty
    \end{align*}

    Así, $\|Ax\|_2\leq \sqrt{\|Ax\|_1 \|Ax\|_\infty}$, en consecuencia $\|A\|_2 \leq \sqrt{\|A\|_1 \|A\|_\infty}$
    \end{proof}
    Coyo
\end{enumerate}
%!TEX root = ../main.tex

\section{Problema 2}
Sea
\[
f(x) = (x - r_1)(x - r_2) \dots (x - r_d),
\]
donde $r_1 < r_2 < \dots < r_d$.

\begin{itemize}
    \item Probar que si $x_0 > r_d$ la sucesión de Newton-Raphson converge a $r_d$.

    \begin{proof}
        Escribamos $f(x)=(x-r_d)g(x)$ con $g(x)=(x-r_1)\cdots (x-r_{d-1}).$ De esta manera, $f'(x)=g(x)+(x-r_d)g'(x)$, luego la iteración del metodo de Newton Raphson es de la forma:
         \[
         x_{k+1}=x_k-\frac{(x_k-r_d)g(x)}{(x_k-r_d)g'(x_k)+g(x_k)}
        \]
        Ahora asuma que $x_0>r_d$, es decir $x_0-r_d > 0$, esto implica que $x_0-r_i>0$ para $i=1,\ldots,a$ pues $r_d$ es la mayor raíz de $f(x)$. Con esto en mente note que:
        \begin{align*}
        x_1 - r_d &= x_0 - r_d - \frac{(x_0-r_d)g(x)}{(x_0-r_d)g'(x_0)+g(x_0)} \\ 
        &=(x_0 - r_d)\left(1-\frac{g(x_0)}{(x_0-r_d)g'(x_0)+g(x_0)}\right)
        \end{align*}
        Ahora, veamos que $g'(x_0)>0$:
        \[
        g'(x_0)=\sum_{i=1}^{d-1}\left(\prod_{\substack{j=1 \\ j \neq i}}^{d-1} (x_0 - r_j)\right)
        \]
        Como cada factor de cada producto es positivo, tenemos que cada sumando es positivo, y por lo tanto $g'(x_0) >0$,luego $(x_0-r_d)g'(x_0)+g(x)>g(x)$. Así
        \[
        0 < \frac{g(x_0)}{(x_0-r_d)g'(x_0)+g(x_0)} <1
        \]
        Y así
        \[
        x_1-r_d=(x_0 - r_d)\left(1-\frac{g(x_0)}{(x_0-r_d)g'(x_0)+g(x_0)}\right) >0
        \]
        Por lo tanto $x_1-r_d > 0$. Procediendo de manera inductiva llegamos a que $x_k - r_d >0$ y además, como
        \[
        0<\left(1-\frac{g(x_{k-1})}{(x_{k-1}-r_d)g'(x_{k-1})+g(x_{k-1})}\right) < 1
        \]
        Vemos que existe una constante $M \in (0,1)$ tal que $x_k - r_d<M(x_{k+1} - r_d)$ lo cual implica que $x_k-r_d < M^{k}(x_0 - r_d)$, como $M^{k}\xrightarrow{n\rightarrow \infty} 0$, tenemos que $x_k$ converge a $r_d$

    \end{proof}
    \item Para un polinomio
    \[
    P(x) = a_d x^d + \dots + a_0, \quad a_d \neq 0,
    \]
    tal que sus $d$ raíces son reales y distintas, se propone el siguiente método para aproximar todas sus raíces:
    \begin{itemize}
        \item Se comienza con un valor $x_0$ mayor que
        \[
        M = \max \left\{ 1, \sum_{i=0}^{d-1} \frac{|a_i|}{|a_d|} \right\}.
        \]
        \item Se genera a partir de $x_0$ la sucesión de Newton-Raphson, que, según el ítem anterior, converge a la raíz más grande de $P$, llamémosla $r_d$; obteniéndose de este modo un valor aproximado $\tilde{r_d}$.
        \item Se divide $P$ por $x - \tilde{r_d}$ y se desprecia el resto, dado que $r_d \approx \tilde{r_d}$. Se redefine ahora $P$ como el resultado de esta división y se comienza nuevamente desde el primer ítem, para hallar las otras raíces.
    \end{itemize}
    \item Aplicar este método para aproximar todas las raíces del polinomio
    \[
    P(x) = 2x^3 - 4x + 1.
    \]
\end{itemize}

%!TEX root = ../main.tex

\section{Problema 3}
Sea $f \in C^2[a, b]$, y sean $x_0 = a, x_1 = a + h, \dots, x_n = b$, donde $h = \dfrac{b-a}{n}$. Considerar la poligonal $l(x)$ que interpola a $f$ en los puntos $x_i, i = 0 \dots n$. Probar que

\begin{itemize}
    \item[a)]$$|f(x) - l(x)| \leq \frac{h^2}{2} \max_{x \in [a,b]} |f''(x)|.$$

    \begin{proof}
        Primero recordemos que $l(x)=\dfrac{f(x_{i+1})-f(x_i)}{h}(x-x_i)+f(x_i)$ para $[x_i,x_{i+1}]$, note  que por el teorema de Taylor

        $$l(x)=f(x_i)+f^{\prime}(x_i)(x-x_i)+\frac{f^{\prime\prime}(c_1)}{2}(x_{i+1}-x_i)(x-x_i), \text{ con } c_1\in (x_i,x),$$

        luego 
        $$f(x)-l(x)=(f(x)-f(x_i))-\left(f^{\prime}(x_i)(x-x_i)+\frac{f^{\prime\prime}(c_1)}{2}(x_{i+1}-x_i)(x-x_i)\right).$$

        Aplicando nuevamente teorema de Taylor

        $$f(x)-f(x_i)=(x-x_i)f^{\prime}(x_i)+\dfrac{f^{\prime\prime}(c_2)}{2}(x-x_i)^2, \text{ con }c_2\in (x_i,x),$$

        ahora juntando las identidades obtenemos que

        \begin{align*}
            f(x)-l(x)&=\frac{f^{\prime\prime}(c_2)}{2}(x-x_i)^2-\frac{f^{\prime\prime}(c_1)}{2}(x-x_i)(x_{i+1}-x_i)\\
            &=\frac{1}{2}(x-x_i)(f^{\prime\prime}(c_2)(x-x_i)-f^{\prime\prime}(c_1)(x_{i+1}-x_i))
        .\end{align*}
        por la continuidad de $f^{\prime\prime}$, existe un $C$ tal que 

        $$f(x)-l(x)=\frac{1}{2}(x-x_i)f^{\prime\prime}(C)((x-x_i)-(x_{i+1}-x_i))=\frac{1}{2}f^{\prime\prime}(C)(x-x_i)(x-x_{i+1}).$$

        Finalmente, para $x\in [x_i,x_{i+1}]$ obtenemos que

        $$|f(x)-l(x)|\leq\frac{1}{2}|f^{\prime\prime}(C)|(x_{i+1}-x_i)^2=\frac{h^2}{2}|f^{\prime\prime}(C)|,$$

        por tanto en el intervalo $[a.b]$ se sigue que

        $$|f(x)-l(x)|\leq=\frac{h^2}{2}\max_{x\in [a,b]}|f^{\prime\prime}(x)|$$
    \end{proof}

    \item[b)] $$|f'(x) - l'(x)| \leq h \max_{x \in [a,b]} |f''(x)|.$$

    \begin{proof}
        Como $l(x)=\dfrac{f(x_{i+1})-f(x_i)}{h}(x-x_i)+f(x_i)=f^{\prime}(C)(x-x_i)+f(x_i)$, tenemos que $l^{\prime}(x)=f^{\prime}(C)$ para algún $C\in (x_i,x_{i+1})$ por el teorema del valor medio, utilizando las identidades del punto anterior, se sigue que

        $$f^{\prime}(x)=f^{\prime}(x_i)(x-x_i)^{\prime}+\frac{f^{\prime\prime}(c_2)}{2}((x-x_i)^2)^{\prime}=f^{\prime}(x_i)+f^{\prime\prime}(c_2)(x-x_i)$$

        y por tanto

        \begin{align*}
            |f^{\prime}(x)-l^{\prime}(x)|&=|f^{\prime}(x_i)+f^{\prime\prime}(c_2)(x-x_i)-f^{\prime}(x_i)-f^{\prime\prime}(c_3)(C-x_i)|\\
            &=|f^{\prime\prime}(c_2)(x-x_i)-f^{\prime\prime}(c_3)(C-x_i)|\\
            &\leq h \max_{x \in [a,b]} |f''(x)|
        .\end{align*}
    \end{proof}
\end{itemize}
%!TEX root = ../main.tex

\item Demuestra que $[0,1]^\omega$ no es localmente compacto en la topología uniforme.
%!TEX root = ../main.tex

\section*{Ejercicio 5}
Considere el sistema $Ax = b$ donde
\[
A = \begin{bmatrix} 3 & -1 & -1 & 0 & 0 \\ -1 & 4 & 0 & -2 & 0 \\ -1 & 0 & 3 & -1 & 0 \\ 0 & -2 & -1 & 5 & -1 \\ 0 & 0 & 0 & -1 & 2 \end{bmatrix}, \quad b = \begin{bmatrix} 2 \\ -26 \\ 3 \\ 47 \\ -10 \end{bmatrix}
\]
\begin{enumerate}
    \item[a)] Investigue la convergencia de los métodos de Jacobi, Gauss-Seidel y sobrerelajación.
    \item[b)] ¿Cuál es el radio espectral de la matriz $J$ y de la matriz $S$?
    \item[c)] Aproxime con dos cifras decimales el parámetro de sobrerelajación $\omega^*$.
    \item[d)] ¿Qué reducción en el costo operacional ofrece el método de sobrerelajación con el parámetro $\omega^*$, en comparación con el método de Gauss-Seidel?
    \item[e)] ¿Cuántas iteraciones más requiere el método de Gauss-Seidel para lograr una precisión mejorada en una cifra decimal? ¿Cuántas necesita el método de sobrerelajación con $\omega^*$?
\end{enumerate}

\end{document}