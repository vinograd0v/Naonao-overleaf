%!TEX root = ../main.tex

\section*{Ejercicio 3}
Sean
\[
A = \begin{bmatrix} a & 0 & 0 \\ a\delta & a & 0 \\ 0 & a\delta & a \end{bmatrix}, \quad a < 0, \delta > 0, \quad b = \begin{bmatrix} -1 \\ -1.1 \\ 0 \end{bmatrix}
\]
\begin{enumerate}
    \item[a)] Obtenga el número de condición de $A$.
    \item[b)] Estudie el condicionamiento del sistema $Ax = b$ en función de los valores de $\delta$. Interprete su resultado.
    \item[c)] Si $a = -1$, $\delta = 0.1$ y se considera $x^* = (1, 9/10, 1)^T$ como solución aproximada del sistema $Ax = b$ (sin obtener la solución exacta), determine un intervalo en el que esté comprendido el error relativo. ¿Es coherente con la respuesta dada en el apartado anterior?
    \item[d)] Si $a = -1$ y $\delta = 0.1$, ¿es convergente el método de Jacobi aplicado a la resolución del sistema $Ax = b$? Realice tres iteraciones a partir de $x_0 = (0, 0, 0)^T$.
\end{enumerate}