%!TEX root = ../main.tex

\subsection*{Ejercicio 4.}
Al aproximar una función continua $f : [0, 1] \to \mathbb{R}$ mediante un polinomio $p(t) = a_n t^n + \cdots + a_1 t + a_0$, el error de aproximación $E$ se mide en la norma $L^2$, es decir:
\[
E^2 := \|p - f\|_{L^2}^2 = \int_0^1 [p(t) - f(t)]^2 \, dt.
\]

\begin{enumerate}
    \item[(a)] Muestre que la minimización del error $E = E(a_0, \ldots, a_n)$ conduce a un sistema de ecuaciones lineales $H_n a = b$, donde:
    \[
    b = [b_0, \ldots, b_n]^T \in \mathbb{R}^{n+1}, \quad b_i = \int_0^1 f(t)t^i \, dt, \quad i = 0, \ldots, n,
    \]
    y $H_n$ es la matriz de Hilbert de orden $n$, definida como:
    \[
    (H_n)_{i,j} = \frac{1}{i + j + 1}, \quad i, j = 0, \ldots, n.
    \]
    El vector $a$ representa los coeficientes del polinomio $p$.\\

    \begin{proof}
    Sea $p(t)=\displaystyle\sum_{j=0}^{n} a_jt^j$, note que 

    \begin{align*}
        E^2(a_0,\ldots,a_n)&=\int_0^1\left(p(t)-f(t)\right)^2dt\\
        &=\int_0^1\left(\sum_{j=0}^{n} a_jt^j-f(t)\right)^2dt\\
        &=\int_0^1\left(\sum_{j=0}^{n} a_jt^j\right)^2dt-2\int_0^1f(t)\sum_{j=0}^{n} a_jt^j dt+\|f\|_{L^2}^2
    .\end{align*}

    Note que esta función depende de los coneficientes, por lo tanto para minimizar el error podemos derivar parcialmente con respecto a $a_k$, $0\leq k\leq n$ e igualar a 0. Derivando obtenemos que

    \begin{align*}
        \frac{\partial E^2}{\partial a_k}=2\int_0^1\left(\sum_{j=0}^{n} a_jt^{k+j}\right)dt-2\int_0^1f(t)t^kdt=0 
    ,\end{align*}
    
    despejando de esta ecuación obtenemos que

    $$\int_0^1\sum_{j=0}^{n} a_jt^{j+k}dt=\sum_{j=0}^{n} a_j\int_0^1t^{j+k}dt=\int_0^1f(t)t^k dt,$$

    más aún, como $\displaystyle\int_0^1t^{k+j}dt=\frac{1}{k+j+1}=(H_n)_{k,j}$, obtenemos

    $$\sum_{j=0}^{n} a_j(H_n)_{k,j}=(H_n)_k\begin{pmatrix}
           a_{1} \\
           a_{2} \\
           \vdots \\
           a_{n}
         \end{pmatrix}=\int_0^1f(t)t^kdt=b_k,$$

donde $(H_n)_k$ denota la $k-$ésima fila de la matriz $(H_n)$, que es lo mismo que $H_n a=b$.


    \end{proof}

    \item[(b)] Muestre que $H_n$ es simétrica y definida positiva.\\

    \begin{proof}
    Note que la simetría es inmediata de  que

    $$(H_n)_{i,j}=\frac{1}{i+j+1}=\frac{1}{j+i+1}=(H_n)_{j,i}.$$

    Para ver que la matriz es definida positiva note que
    $$
\begin{aligned}
x^T H_n x & =\left(x_1, \ldots, x_n\right)\left(\begin{array}{ccc}
1 & \cdots & \dfrac{1}{n+2} \\
\vdots & \ddots & \vdots \\
\dfrac{1}{n+2} & \cdots & \dfrac{1}{2 n+1}
\end{array}\right)\left(\begin{array}{c}
x_1 \\
\vdots \\
x_n
\end{array}\right) \\
& =\sum_{i=0}^n \sum_{j=0}^n \frac{x_j x_i}{i+j+1} \\
& =\sum_{i=0}^n \sum_{j=0}^n x_j x_i \int_0^1 t^i t^j d t \\
& =\int_0^1 \sum_{i=0}^n x_i t^i \sum_{j=0}^n x_j t^j d t \\
& =\left\|\sum_{j=0}^{n} x_jt^j\right\|_{L^2}^2>0
\end{aligned}
$$
    \end{proof}

    \item[(c)] Solucione el sistema $H_n x = b$, donde $b$ tiene componentes $b_i = 1 / (n + i +1)$, para $i = 0, \ldots, n$. Para esto, use las factorizaciones LU ($[L, U] = \text{lu}(H)$) y Cholesky ($L = \text{chol}(H)$). Luego resuelva los dos sistemas triangulares.

    \begin{solution}
    El teorema de Stone-Weierstrass nos dice que los polinomios son densos en las funciones continuas, esto nos da que la aproximación que nos piden es posible, al truncar el número de coeficientes del polinomio vimos que el problema de optimizarlos se reduce a resolver el sistema $H_nx=b$, esto podemos programarlo en Matlab, Python, etc. Para este caso nosotros hemos hecho el trabajo en ambos lenguajes con el propósito de comparar resultados, antes de presentar los códigos en Matlab observemos que

    $$b_i=\dfrac{1}{n+i+1}=\int_0^1f(t)t^i dt,$$

    y por lo tanto $f(t)=t^n$, en efecto estamos aproximando el polinomio $t^n$ por polinomios, así pues, esperaríamos una solución numérica del estilo $a=(0,\ldots,1)^T$. Para el caso de la factorización $LU$ implementamos

    \begin{lstlisting}
    n = 10; 
    H = hilb(n); %Genera la matriz de Hilbert de orden n
    b = zeros(n, 1); %Crea un vector con n ceros 
    for i = 1:n
        b(i) = 1 / (i + n - 1); %Cambia  las entradas por las del ejercicio
    end
    [L, U] = lu(H);

    %Solucionamos los sistemas

    y = L \ b;
    a_LU = U \ y;

    disp(a_LU')
    \end{lstlisting}

    En el caso de la factorización de Cholesky

    \begin{lstlisting}
    n = 10; 
    H = hilb(n); 
    b = zeros(n, 1); 
    for i = 1:n
        b(i) = 1 / (i + n - 1); 
    end
    L = chol(H); 
    y = L' \ b;
    a_LU = L \ y;
    disp(a_LU')
    \end{lstlisting}

    En ambos casos tomamos un tamaño $n=10$ para probar el algoritmo, en donde la factorización $LU$ nos arrojó el resultado $x=(0,0,0,0,0,0,0,0,0,1)^T$ y la factorización de Cholesky nos dió el resultado 

   \[x=
\begin{bmatrix}
4.4438 \times e^{-11} \\
-3.8598 \times e^{-9} \\
8.2505 \times e^{-8} \\
-7.5194 \times e^{-7} \\
3.5931 \times e^{-6} \\
-9.8904 \times e^{-6} \\
1.6243 \times e^{-5} \\
-1.5708 \times e^{-5} \\
8.2514 \times e^{-6} \\
1
\end{bmatrix}
\]
    \end{solution}

    \item[(d)] Para ambos métodos, ¿qué tan precisas son las soluciones numéricas $\hat{x}_{\text{approx}}$? Tabule los errores de la solución:
    \[
    e(n) = \|\hat{x}_{\text{approx}} - x_{\text{exact}}\|
    \]
    como una función de $n = 2, \ldots, 15$. Note que $x_{\text{exact}} = (0, \ldots, 1)^T$. Puede graficar los errores en función de $n$ utilizando la función \texttt{semilogy} de Matlab. Explique en detalle los resultados.
\end{enumerate}

jmm