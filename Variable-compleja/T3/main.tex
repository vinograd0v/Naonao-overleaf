\documentclass[11pt]{article}
\usepackage[spanish]{babel}
\usepackage[T1]{fontenc}
\usepackage{amsmath,amsfonts,amssymb,amsthm}
\usepackage{amsfonts}
\usepackage{graphicx}
\usepackage{geometry}
\usepackage{newpxtext,euler}
\usepackage{float}
\usepackage{xcolor}
\usepackage{geometry}
 \geometry{
 a4paper,
 total={170mm,245mm},
 left=20mm,
 top=30mm,
 }
\pagestyle{empty}

\title{Taller II}

\author{Bourbaki}
\date{\today}


\begin{document}

\maketitle


\begin{enumerate}
    \item Demuestre que \( A \subset \mathbb{C} \) es compacto si y solo si es acotado y cerrado.

    \item Sea \( K \subset \mathbb{C} \) compacto. Sean \( K_1 \supseteq K_2 \supseteq K_3 \supseteq \cdots \) una sucesión de subconjuntos de \( K \) no vacíos, tales que \( K_n \supseteq K_{n+1} \). Demostrar que la intersección de todos los \( K_n, n = 1, 2, 3, \ldots \) es no vacía.

    \item (*) Encontrar la imagen de las regiones:
    \[
    1 < | \operatorname{Im}(z) | \leq 2
    \]
    \[
    |z| < 1
    \]
    bajo las aplicaciones:
    \begin{enumerate}
        \item \( f(z) = z^2 \)
        \item \( f(z) = \dfrac{2z + i}{z + 1} \)
    \end{enumerate}

    \item (*) Sea \( f(z) = \dfrac{z - i}{z + i} \), hallar la imagen por \( f \) de:
    \begin{enumerate}
        \item El semiplano superior.
        \item La semirecta \( it; t \geq 0 \).
        \item La recta \( it; t \in \mathbb{R} \).
        \item \( |z - 1| = 1 \).
        \item \( |z| = 2; \operatorname{Im}(z) \geq 0 \).
    \end{enumerate}

    \item Sea \( A = \{ z \in \mathbb{C} : -\infty < \operatorname{Im}(z) \leq \alpha \} \). Si \( f(z) = e^z \), hallar \( f(A) \).

    \item Sea \( A = \{ z \in \mathbb{C} : |\operatorname{Re}(z)| < \frac{\pi}{2}, \operatorname{Im}(z) > 0 \} \). Si \( f(z) = \sin(z) \), hallar \( f(A) \).

    \item Determine completamente la proyección estereográfica (que lleva la esfera de Riemann en el plano complejo). Es decir, hallar explícitamente \( T \) y \( T^{-1} \).

    \item Demostrar que la proyección estereográfica preserva círculos. La imagen directa o inversa de una circunferencia es una circunferencia.
\end{enumerate}

\end{document}