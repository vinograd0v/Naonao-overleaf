%!TEX root = ../main.tex

\item Sea $A$ un subespacio cerrado de $X$. Demuestre que si $X$ es Lindelöf, entonces $A$ es Lindelöf.  
Muestre con un ejemplo que si $X$ tiene un subconjunto denso numerable, $A$ no necesariamente tiene un subconjunto denso numerable.

\begin{proof}
    Sea $\mathcal{C}$ un cubrimiento de $A$, como $A$ es cerrado entonces $\mathcal{C}\cup A^c$ es un cubrimiento  de $X$, como $X$ es Lindelöf entonces tiene un subcubrimiento contable, digamos 

    $$\bigcup_{i=1}^{\infty}C_i\cup A^{C}=X, 
        $$
     con $C_i\in \mathcal{C}$, luego $A=\bigcup_{i=1}^{\infty}C_i$
\end{proof}

El ejemplo en cuestión es el plano de Sorgenfrey: Como el plano es producto finito de espacios separables, es separabe, ahora considere la recta $L$ dada por la ecuación $y=-x$, luego, si consideramos el abierto $U=[a,1)\times [-a,1)$, vemos que $U \cap L = \{a\}$, por tanto los puntos son abiertos en $L$ y así, $L$ tiene la topología discreta, por tanto no tiene un subconjunto denso contable.