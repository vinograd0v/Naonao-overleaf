\documentclass[11pt]{article}
\usepackage[spanish]{babel}
\usepackage[T1]{fontenc}
\usepackage{amsmath,amsfonts,amssymb,amsthm}
\usepackage{amsfonts}
\usepackage{graphicx}
\usepackage{amssymb}
\usepackage{geometry}
\usepackage{newpxtext,euler}
\usepackage{xcolor}
\usepackage{geometry}
\usepackage{picinpar}
 \geometry{
 a4paper,
 total={170mm,245mm},
 left=20mm,
 top=30mm,
 }
\pagestyle{empty}
\begin{document}
\setlength{\parindent}{0cm}
\hoffset-0.46cm
\voffset-1.46cm
\begin{window}[0,l,{\includegraphics[scale=0.31]{Graphics/logo.eps}},]
\large\scshape \hspace{1.4cm}\textsf{Universidad Nacional de Colombia} \\
\textcolor{white}{\tiny.}  \large \hspace{2.4cm} \textsf{Facultad de Ciencias} \\
\textcolor{white}{\tiny.}   \normalsize\hspace{2.8cm}\textsf{Topología General}\\
\end{window}
\vspace{0.6cm}
\small
\textsf{Mateo Andrés Manosalva Amaris\\
Sergio Alejandro Bello Torres} 
\normalsize
\dotfill
\vspace{0.7cm}


\begin{enumerate}
    \item Sean $\mathcal{T}$ y $\mathcal{T}'$ dos topologías sobre $X$. Si $\mathcal{T}' \supset \mathcal{T}$, ¿qué implica la conexidad de $X$ en una topología sobre la otra?
    
    \item Sea $\{A_n\}$ una secuencia de subespacios conexos de $X$, tal que $A_n \cap A_{n+1} \neq \emptyset$ para todo $n$. Demuestra que $\bigcup A_n$ es conexo.
    
    \item Sea $\{A_\alpha\}$ una colección de subespacios conexos de $X$; sea $A$ un subespacio conexo de $X$. Muestra que si $A \cap A_\alpha \neq \emptyset$ para todo $\alpha$, entonces $A \cup \big(\bigcup A_\alpha\big)$ es conexo.
    
    \item Demuestra que si $X$ es un conjunto infinito, entonces es conexo en la topología del complemento finito.
    
    \item Un espacio es \textit{totalmente disconexo} si sus únicos subespacios conexos son conjuntos de un solo punto. Muestra que si $X$ tiene la topología discreta, entonces $X$ es totalmente desconexo. ¿Es cierto el recíproco?
    
    \item Sea $A \subset X$. Muestra que si $C$ es un subespacio conexo de $X$ que intersecta tanto $A$ como $X - A$, entonces $C$ intersecta $\mathrm{Bd} A$.
    
    \item ¿Es el espacio $\mathbb{R}_\ell$ conexo? Justifica tu respuesta.
    
    \item Determina si $\mathbb{R}^\omega$ es conexo en la topología uniforme.
    
    \item Sea $A$ un subconjunto propio de $X$, y sea $B$ un subconjunto propio de $Y$. Si $X$ e $Y$ son conexos, muestra que
    \[
    (X \times Y) - (A \times B)
    \]
    es conexo.

    \item Sea $\{X_\alpha\}_{\alpha \in J}$ una familia indexada de espacios conexos; sea $X$ el espacio producto
    \[
    X = \prod_{\alpha \in J} X_\alpha.
    \]
    Sea $\mathbf{a} = (a_\alpha)$ un punto fijo de $X$.
    \begin{enumerate}
        \item Dado cualquier subconjunto finito $K$ de $J$, sea $X_K$ el subespacio de $X$ que consiste en todos los puntos $\mathbf{x} = (x_\alpha)$ tales que $x_\alpha = a_\alpha$ para $\alpha \notin K$. Muestra que $X_K$ es conexo.
        
        \item Demuestra que la unión $Y$ de los espacios $X_K$ es conexa.
        
        \item Demuestra que $X$ es igual a la clausura de $Y$; concluye que $X$ es conexo.
    \end{enumerate}
    
    \item Sea $p : X \to Y$ un mapeo cociente. Demuestra que si cada conjunto $p^{-1}(\{y\})$ es conexo, y si $Y$ es conexo, entonces $X$ es conexo.
    
    \item Sea $Y \subset X$; sean $X$ e $Y$ conexos. Demuestra que si $A$ y $B$ forman una separación de $X - Y$, entonces $Y \cup A$ y $Y \cup B$ son conexos.

\end{enumerate}
   


\end{document}