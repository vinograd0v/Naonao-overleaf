%!TEX root = ../main.tex

\subsection*{Ejercicio 2.}
Sea $\| \cdot \|$ una norma en $\mathbb{R}^n$ y $A$ una matriz invertible de tamaño $n \times n$. Pruebe que:\\

Si $Ax = b$, $(A + \delta A)(x + \delta x) = b + \delta b$ y $\|A^{-1}\| \|\delta A\| < 1$, entonces $A + \delta A$ es invertible y se cumple que:
\[
\frac{\|\delta x\|}{\|x\|} \leq \frac{\text{cond}(A)}{1 - \|A^{-1}\| \|\delta A\|} \left( \frac{\|\delta A\|}{\|A\|} + \frac{\|\delta b\|}{\|b\|} \right).
\]

\begin{sproof}
    Primero  mostremos que $A+\delta A$ es invertible. Notemos que como $A$ es invertible tenemos

     $$A+\delta A=A(I-(-A^{-1}\delta A)).$$

     Luuego por hipotesis y usando que la norma matricial es submultiplicativa tenemos que
     $$\|-A^{-1}\delta A\|\leq\|A^{-1}\|\|\delta A\|<1.$$
     Por lo que $I+A^{-1}\delta A$ es invertible por el teorema de la serie de Neumann. Asi $A+\delta A$ es invertible, pues es el producto de dos matrices invertibles.\\

     Ahora probemos la desigualdad. Como por hipotesis $Ax=b$, notemos que 
     \begin{align*}
     (A+\delta A)(x+\delta x)&=Ax+A\delta x+\delta Ax+\delta A\delta x\\
     &=b+A\delta x+\delta Ax+\delta A\delta x.
     \end{align*}

     Reemplazando en la hipotesis tenemos que
     $$b+A\delta x+\delta Ax+\delta A\delta x=b+\delta b,$$
     luego
     $$A\delta x=\delta b-\delta Ax-\delta A\delta x,$$
     y como $A$ es invertible
     $$\delta x=A^{-1}\delta b-A^{-1}\delta Ax-A^{-1}\delta A\delta x.$$
     Si tomamos la norma en $\mathbb{R}^n$ en ambos lados, por la desigualdad triangular y el hecho de que la norma matricial inducida es compatible y submultiplicativa tenemos que
     \begin{align*}
     \|\delta x\|&=\|A^{-1}\delta b-A^{-1}\delta Ax-A^{-1}\delta A\delta x\|\\
     &\leq \|A^{-1}\delta b\|+\|A^{-1}\delta Ax\|+\|A^{-1}\delta A\delta x\|\\
     &\leq \|A^{-1}\|\|\delta b\|+\|A^{-1}\|\|\delta A\|\|x\|+\|A^{-1}\|\|\delta A\|\|\delta x\|.
     \end{align*}
     Luego restando el ultimo sumando y factorizando $\|\delta x\|$ tenemos
     $$\|\delta x\|(1-\|A^{-1}\|\|\delta A\|)\leq \|A^{-1}\|\|\delta b\|+\|A^{-1}\|\|\delta A\|\|x\|. $$
     Como $A$ es invertible, es distinta de 0, luego $\|A\|>0$, asi tenemos que por la definicion del numero de condicion $\|A^{-1}\|=\dfrac{cond(A)}{\|A\|}.$ Ademas tenemos que $\|b\|=\|Ax\|\leq\|A\|\|x\|,$ por tanto $\dfrac{1}{\|A\|\|x\|}\leq\dfrac{1}{\|b\|}.$ Con todo esto notemos que para el lado derecho de la desigualdad tenemos que
     \begin{align*}
       \|A^{-1}\|\|\delta b\|+\|A^{-1}\|\|\delta A\|\|x\|&=\dfrac{cond(A)}{\|A\|\|x\|}\|\delta b\|\|x\|+\dfrac{cond(A)}{\|A\|}\|\delta A\|\|x\|\\
       &\leq\dfrac{cond(A)}{\|b\|}\|\delta b\|\|x\|+\dfrac{cond(A)}{\|A\|}\|\delta A\|\|x\|\\
       &=cond(A)\|x\|\left(\dfrac{\|\delta A\|}{\|A\|}+\dfrac{\|\delta b\|}{\|b\|}\right).
     \end{align*}
     Por lo tanto tenemos que
     $$\|\delta x\|(1-\|A^{-1}\|\|\delta A\|)\leq cond(A)\|x\|\left(\dfrac{\|\delta A\|}{\|A\|}+\dfrac{\|\delta b\|}{\|b\|}\right).$$
     Como por hipotesis $0<1-\|A^{-1}\|\|\delta A\|$, podemos dividir y mantener la desigualdad, asi dividiento tambien por $\|x\|$ concluimos que
     $$\frac{\|\delta x\|}{\|x\|} \leq \frac{\text{cond}(A)}{1 - \|A^{-1}\| \|\delta A\|} \left( \frac{\|\delta A\|}{\|A\|} + \frac{\|\delta b\|}{\|b\|} \right).$$ 
    



\end{sproof}