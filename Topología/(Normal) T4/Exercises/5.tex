%!TEX root = ../main.tex

\item La condición de Hausdorff es una parte esencial en la definición de una variedad; no se sigue de las otras partes de la definición. Considere el siguiente espacio:  
    Sea \( X \) la unión del conjunto \( \mathbb{R} - \{ 0 \} \) y el conjunto de dos puntos \( \{ p, q \} \). Se topologiza \( X \) tomando como base la colección de todos los intervalos abiertos en \( \mathbb{R} \) que no contienen \( 0 \), junto con todos los conjuntos de la forma \( (-a,0) \cup \{ p \} \cup (0,a) \) y todos los conjuntos de la forma \( (-a,0) \cup \{ q \} \cup (0,a) \), para \( a > 0 \). El espacio \( X \) se llama \textit{la recta con dos orígenes}.
    
    \begin{enumerate}
        \item Verificar que esto es una base para una topología.
        \item Mostrar que cada uno de los espacios \( X - \{ p \} \) y \( X - \{ q \} \) es homeomorfo a \( \mathbb{R} \).
        \item Mostrar que \( X \) satisface el axioma \( T_1 \), pero no es de Hausdorff.
        \item Mostrar que \( X \) satisface todas las condiciones para ser una 1-variedad, excepto la condición de Hausdorff.
    \end{enumerate}