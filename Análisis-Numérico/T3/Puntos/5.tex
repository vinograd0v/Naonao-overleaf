%!TEX root = ../main.tex

\section*{Ejercicio 5}
Considere el sistema $Ax = b$ donde
\[
A = \begin{bmatrix} 3 & -1 & -1 & 0 & 0 \\ -1 & 4 & 0 & -2 & 0 \\ -1 & 0 & 3 & -1 & 0 \\ 0 & -2 & -1 & 5 & -1 \\ 0 & 0 & 0 & -1 & 2 \end{bmatrix}, \quad b = \begin{bmatrix} 2 \\ -26 \\ 3 \\ 47 \\ -10 \end{bmatrix}
\]
\begin{enumerate}
    \item[a)] Investigue la convergencia de los métodos de Jacobi, Gauss-Seidel y sobrerelajación.
    \item[b)] ¿Cuál es el radio espectral de la matriz $J$ y de la matriz $S$?
    \item[c)] Aproxime con dos cifras decimales el parámetro de sobrerelajación $\omega^*$.
    \item[d)] ¿Qué reducción en el costo operacional ofrece el método de sobrerelajación con el parámetro $\omega^*$, en comparación con el método de Gauss-Seidel?
    \item[e)] ¿Cuántas iteraciones más requiere el método de Gauss-Seidel para lograr una precisión mejorada en una cifra decimal? ¿Cuántas necesita el método de sobrerelajación con $\omega^*$?
\end{enumerate}