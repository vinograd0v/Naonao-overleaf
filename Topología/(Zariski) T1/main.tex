\documentclass[11pt]{article}
\usepackage[spanish]{babel}
\usepackage[T1]{fontenc}
\usepackage{amsmath,amsfonts,amssymb,amsthm}
\usepackage{amsfonts}
\usepackage{graphicx}
\usepackage{amssymb}
\usepackage{geometry}
\usepackage{newpxtext}
\usepackage{xcolor}
\usepackage{geometry}
\usepackage{picinpar}
\usepackage{accents}
\newcommand{\interior}[1]{\accentset{\smash{\raisebox{-0.12ex}{$\scriptstyle\circ$}}}{#1}\rule{0pt}{2.3ex}}
\fboxrule0.0001pt \fboxsep0pt
 \geometry{
 a4paper,
 total={170mm,245mm},
 left=20mm,
 top=30mm,
 }
\pagestyle{empty}
\begin{document}
\setlength{\parindent}{0cm}
\hoffset-0.46cm
\voffset-1.46cm
\begin{window}[0,l,{\includegraphics[scale=0.31]{Graphics/logo.eps}},]
\large\scshape \hspace{1.4cm}\textsf{Universidad Nacional de Colombia} \\
\textcolor{white}{\tiny.}  \large \hspace{2.4cm} \textsf{Facultad de Ciencias} \\
\textcolor{white}{\tiny.}   \normalsize\hspace{2.8cm}\textsf{Topología General}\\
\end{window}
\vspace{0.6cm}
\small
\textsf{Mateo Andrés Manosalva Amaris\\
Sergio Alejandro Bello Torres} 
\normalsize
\dotfill
\vspace{0.7cm}


\begin{enumerate}
    \item Sean $\tau$ y $\tau'$ dos topologías sobre $X$. Si $\tau' \supset \tau$, ¿qué implica la conexidad de $X$ en una topología sobre la otra?\\

    \textbf{Solución:} Note que si $X$ es conexo en la topología $\tau^{\prime}$, entonces en la topología $\tau$ también lo es. Supongamos que no, entonces existen dos abiertos disjuntos $A$ y $B$ tales que $X=A\cup B$, como $\tau\subset \tau^{\prime}$ entonces $A,B\in \tau^{\prime}$, luego $X$ no sería conexo.\\

    La contrarrecíproca nos dice que si $X$ es disconexo en $\tau$ entonces es disconexo en $\tau^{\prime}$, sin embargo que $X$ sea conexo en $\tau$ no implica conexidad  en la topología $\tau^{\prime}$. Por ejemplo, considere los espacios topológicos $(\mathbb{R},\tau)$, $(\mathbb{R},\tau_{\ell})$, con $\tau$ la topología usual, es claro que $\tau\subset \tau_{\ell}$. Sabemos que $\mathbb{R}$ es conexo en la topología usual, pero $\mathbb{R}_{\ell}$ no lo es. La prueba de esto se encuentra en el ejercicio 7.
    
    \item Sea $\{A_n\}$ una secuencia de subespacios conexos de $X$, tal que $A_n \cap A_{n+1} \neq \emptyset$ para todo $n$. Demuestra que $\bigcup A_n$ es conexo.

    \begin{proof}
        Supongamos que no, esto es

        $$\bigcup_{n}A_n=B\cup C 
            $$

        con $B\cap C=\emptyset$ y $B,C\neq \emptyset$. Tomemos $A_1\subset B$, en efecto
        
        $$I:=\{ i\in\mathbb{N}: A_i\subset C\}\neq \emptyset,$$

        de lo contrario $C=\emptyset$ y esto no es posible. El principio del buen orden garantiza que $I$ tiene un elemento mínimo, digamos $k$, esto nos da que $A_{k-1}\subset B$, así $A_k\cap A_{k-1}=\emptyset$, una contradicción. 
    \end{proof}
    
    \item Sea $\{A_\alpha\}$ una colección de subespacios conexos de $X$; sea $A$ un subespacio conexo de $X$. Muestra que si $A \cap A_\alpha \neq \emptyset$ para todo $\alpha$, entonces $A \cup \big(\bigcup A_\alpha\big)$ es conexo.

    \begin{proof}
        Note que $$A\cup \bigcup_{\alpha}A_{\alpha}=\bigcup_{\alpha}(A\cup A_{\alpha})$$

        y como $A\subset \displaystyle \bigcap_{\alpha}(A\cup A_{\alpha}) 
            $, y $A\neq \emptyset$, entonces por el punto anterior se concluye lo deseado.
    \end{proof}
    
    \item Demuestra que si $X$ es un conjunto infinito, entonces es conexo en la topología del complemento finito.

    \begin{proof}
        Suponga que no, entonces $X=A\cup B$ con $A,B$ abiertos disjuntos, como $A$ y $B$ son disjuntos tenemos que $B=A^{c}$, entonces $B$ es finito ya que $A\in \tau$, como $B\in \tau$ y $A=B^{c}$, se sigue que $A$ es finito, lo que contradice que $X$ es infinito.
    \end{proof}
    
    \item Un espacio es \textit{totalmente disconexo} si sus únicos subespacios conexos son conjuntos de un solo punto. Muestra que si $X$ tiene la topología discreta, entonces $X$ es totalmente disconexo. ¿Es cierto el recíproco?

    \begin{proof}
        En efecto $A=\{x\}_{x\in X}$ es conexo ya que no pueden haber dos abiertos disjuntos no vacíos cuya unión sea $\{x\}$. Si $|A|>2$, note que 

        $$A=\bigcup_{x\in A}\{x\}
            $$

        y por tanto $A$ no es conexo, ya que los singletones son abiertos disjuntos en la topología discreta.\\
    \end{proof}
    
    El recíproco no es cierto. $\mathbb{Q}$ no es conexo con la topología usual y los únicos subespacios conexos de $\mathbb{Q}$ son los conjuntos de un solo punto\\

    Si $Y$ es un subespacio de $\mathbb{Q}$ que contiene dos puntos $p$ y $q$, se puede elegir un número irracional $a$ entre $p$ y $q$, tal que

$$
Y=(Y \cap (-\infty, a)) \cup  (Y \cap (a, +\infty))
$$

y la topología usual  no es la misma topología discreta xd.

    \item Sea $A \subset X$. Muestra que si $C$ es un subespacio conexo de $X$ que intersecta tanto $A$ como $X - A$, entonces $C$ intersecta $\partial A$.

    \begin{proof}
        Suponga que $C\cap \partial A=\emptyset$, como $\overline{A}=\interior{A}\cup \partial A$ entonces $C\cap A\subset \interior{A}$, análogamente $\overline{X-A}=\interior{(X-A)}\cup \partial (X-A)$  y $C\cap (X-A)\subset \interior{(X-A)}$, además

        $$C=(C\cap A) \cup (C\cap (X-A))\subset \interior{(X-A)}\cup \interior{A},$$

        como $C$ es conexo, entonces $C$ cae enteramente en $\interior{(X-A)}$ o en $\interior{A}$, esto contradice que $C\cap (X-A)$ y $C\cap A$.
    \end{proof}
    
    \item ¿Es el espacio $\mathbb{R}_\ell$ conexo? Justifica tu respuesta.\\
    
    \textbf{Falso,} en efecto 

    $$\mathbb{R}_{\ell}=(-\infty,0)\cup [0,\infty)$$
     y esta es una disconexión.

    \item Determina si $\mathbb{R}^\omega$ es conexo en la topología uniforme.


Sean $A$, $B$ $\subset\mathbb{R}^\omega$ los conjuntos de todas las sucesiones acotadas y  no acotadas respectivamente. En efecto $A \cup B = \mathbb{R}^\omega$ y $A \cap B = \emptyset$,  nos falta ver que $A$ y $B$ son abiertos. Si $a \in \mathbb{R}^\omega$, la bola $B(a, \varepsilon)$ si $\varepsilon < 1$ está totalmente contenida en

$$
 \left(a_1 - 1, a_1 + 1\right) \times \cdots \times \left(a_n - 1, a_n + 1\right) \times \cdots
$$

\textcolor{red}{Creo que no entendí bien esa mondá pana}

    
    \item Sea $A$ un subconjunto propio de $X$, y sea $B$ un subconjunto propio de $Y$. Si $X$ e $Y$ son conexos, muestra que
    \[
    (X \times Y) - (A \times B)
    \]
    es conexo.

    \item Sea $\{X_\alpha\}_{\alpha \in J}$ una familia indexada de espacios conexos; sea $X$ el espacio producto
    \[
    X = \prod_{\alpha \in J} X_\alpha.
    \]
    Sea $\mathbf{a} = (a_\alpha)$ un punto fijo de $X$.
    \begin{enumerate}
        \item Dado cualquier subconjunto finito $K$ de $J$, sea $X_K$ el subespacio de $X$ que consiste en todos los puntos $\mathbf{x} = (x_\alpha)$ tales que $x_\alpha = a_\alpha$ para $\alpha \notin K$. Muestra que $X_K$ es conexo.
        
        \item Demuestra que la unión $Y$ de los espacios $X_K$ es conexa.
        
        \item Demuestra que $X$ es igual a la clausura de $Y$; concluye que $X$ es conexo.
    \end{enumerate}
    
    \item Sea $p : X \to Y$ un mapeo cociente. Demuestra que si cada conjunto $p^{-1}(\{y\})$ es conexo, y si $Y$ es conexo, entonces $X$ es conexo.
    
    \item Sea $Y \subset X$; sean $X$ e $Y$ conexos. Demuestra que si $A$ y $B$ forman una separación de $X - Y$, entonces $Y \cup A$ y $Y \cup B$ son conexos.

\end{enumerate}
   


\end{document}