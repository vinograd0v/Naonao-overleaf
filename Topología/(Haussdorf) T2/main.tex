\documentclass[11pt]{article}
\usepackage[spanish]{babel}
\usepackage[T1]{fontenc}
\usepackage{amsmath,amsfonts,amssymb,amsthm}
\usepackage{amsfonts}
\usepackage{graphicx}
\usepackage{amssymb}
\usepackage{geometry}
\usepackage{newpxtext}
\usepackage{xcolor}
\usepackage{geometry}
\usepackage{picinpar}
\usepackage{accents}
\newcommand{\interior}[1]{\accentset{\smash{\raisebox{-0.12ex}{$\scriptstyle\circ$}}}{#1}\rule{0pt}{2.3ex}}
\fboxrule0.0001pt \fboxsep0pt
 \geometry{
 a4paper,
 total={170mm,245mm},
 left=20mm,
 top=30mm,
 }
\pagestyle{empty}
\begin{document}
\setlength{\parindent}{0cm}
\hoffset-0.46cm
\voffset-1.46cm
\begin{window}[0,l,{\includegraphics[scale=0.31]{Graphics/logo.eps}},]
\large\scshape \hspace{1.4cm}\textsf{Universidad Nacional de Colombia} \\
\textcolor{white}{\tiny.}  \large \hspace{2.4cm} \textsf{Facultad de Ciencias} \\
\textcolor{white}{\tiny.}   \normalsize\hspace{2.8cm}\textsf{Topología General}\\
\hspace*{3.9cm}\textsf{Taller II}\\
\end{window}
\vspace{0.6cm}
\textsf{Mateo Andrés Manosalva Amaris\\
Sergio Alejandro Bello Torres} 
\normalsize
\dotfill
\vspace{0.7cm}


\begin{enumerate}
    \item Muestra que los números racionales $\mathbb{Q}$ no son localmente compactos.

    \item Sea $\{X_\alpha\}$ una familia indexada de espacios no vacíos.
    \begin{enumerate}
        \item Demuestra que si $\prod X_\alpha$ es localmente compacto, entonces cada $X_\alpha$ es localmente compacto y $X_\alpha$ es compacto para todos los valores de $\alpha$, salvo un número finito.
        \item Prueba el recíproco, asumiendo el teorema de Tychonoff.
    \end{enumerate}

    \item Sea $X$ un espacio localmente compacto. Si $f: X \rightarrow Y$ es continua, ¿se sigue que $f(X)$ es localmente compacto? ¿Qué ocurre si $f$ es continua y abierta? Justifica tu respuesta.

    \item Demuestra que $[0,1]^\omega$ no es localmente compacto en la topología uniforme.

    \item Si $f: X_1 \rightarrow X_2$ es un homeomorfismo entre espacios Hausdorff localmente compactos, demuestra que $f$ se extiende a un homeomorfismo de sus compactificaciones por un punto.

    \item Demuestra que la compactificación por un punto de $\mathbb{R}$ es homeomorfa al círculo $S^1$.

    \item Demuestra que la compactificación por un punto de $S_{\Omega}$ es homeomorfa a $\bar{S}_{\Omega}$.

    \item Demuestra que la compactificación por un punto de $\mathbb{Z}_{+}$ es homeomorfa al subespacio $\{0\} \cup \{1 / n \mid n \in \mathbb{Z}_{+}\}$ de $\mathbb{R}$.

    \item Demuestra que si $G$ es un grupo topológico localmente compacto y $H$ es un subgrupo, entonces $G / H$ es localmente compacto.

    \item Demuestra que si $X$ es un espacio de Hausdorff localmente compacto en el punto $x$, entonces para cada vecindad $U$ de $x$, existe una vecindad $V$ de $x$ tal que $\bar{V}$ es compacto y $\bar{V} \subset U$.
\end{enumerate}



\end{document}