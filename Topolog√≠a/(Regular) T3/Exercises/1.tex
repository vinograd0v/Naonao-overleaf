%!TEX root = ../main.tex

\item 
    \begin{enumerate}
        \item Un \textit{$G_{\delta}$-conjunto} en un espacio $X$ es un conjunto $A$ que es igual a una intersección numerable de conjuntos abiertos de $X$. Demuestra que en un espacio $T_1$ de primera numerabilidad, cada conjunto unitario es un $G_{\delta}$-conjunto.
        \item Existe un espacio familiar en el cual cada conjunto unitario es un $G_{\delta}$-conjunto, pero que no satisface el axioma de primera numerabilidad. ¿Cuál es?  
        
        La terminología proviene del alemán. La ``G'' representa ``Gebiet,'' que significa ``conjunto abierto,'' y la ``$\delta$'' representa ``Durchschnitt,'' que significa ``intersección.''
    \end{enumerate}