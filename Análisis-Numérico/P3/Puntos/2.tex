%!TEX root = ../main.tex

\section{Problema 2}
Sea
\[
f(x) = (x - r_1)(x - r_2) \dots (x - r_d),
\]
donde $r_1 < r_2 < \dots < r_d$.

\begin{itemize}
    \item Probar que si $x_0 > r_d$ la sucesión de Newton-Raphson converge a $r_d$.

    \begin{proof}
        Escribamos $f(x)=(x-r_d)g(x)$ con $g(x)=(x-r_1)\cdots (x-r_{d-1}).$ De esta manera, $f'(x)=g(x)+(x-r_d)g'(x)$, luego la iteración del metodo de Newton Raphson es de la forma:
         \[
         x_{k+1}=x_k-\frac{(x_k-r_d)g(x)}{(x_k-r_d)g'(x_k)+g(x_k)}
        \]
        Ahora asuma que $x_0>r_d$, es decir $x_0-r_d > 0$, esto implica que $x_0-r_i>0$ para $i=1,\ldots,a$ pues $r_d$ es la mayor raíz de $f(x)$. Con esto en mente note que:
        \begin{align*}
        x_1 - r_d &= x_0 - r_d - \frac{(x_0-r_d)g(x)}{(x_0-r_d)g'(x_0)+g(x_0)} \\ 
        &=(x_0 - r_d)\left(1-\frac{g(x_0)}{(x_0-r_d)g'(x_0)+g(x_0)}\right)
        \end{align*}
        Ahora, veamos que $g'(x_0)>0$:
        \[
        g'(x_0)=\sum_{i=1}^{d-1}\left(\prod_{\substack{j=1 \\ j \neq i}}^{d-1} (x_0 - r_j)\right)
        \]
        Como cada factor de cada producto es positivo, tenemos que cada sumando es positivo, y por lo tanto $g'(x_0) >0$,luego $(x_0-r_d)g'(x_0)+g(x)>g(x)$. Así
        \[
        0 < \frac{g(x_0)}{(x_0-r_d)g'(x_0)+g(x_0)} <1
        \]
        Y así
        \[
        x_1-r_d=(x_0 - r_d)\left(1-\frac{g(x_0)}{(x_0-r_d)g'(x_0)+g(x_0)}\right) >0
        \]
        Por lo tanto $x_1-r_d > 0$. Procediendo de manera inductiva llegamos a que $x_k - r_d >0$ y además, como
        \[
        0<\left(1-\frac{g(x_{k-1})}{(x_{k-1}-r_d)g'(x_{k-1})+g(x_{k-1})}\right) < 1
        \]
        Vemos que existe una constante $M \in (0,1)$ tal que $x_k - r_d<M(x_{k+1} - r_d)$ lo cual implica que $x_k-r_d < M^{k}(x_0 - r_d)$, como $M^{k}\xrightarrow{n\rightarrow \infty} 0$, tenemos que $x_k$ converge a $r_d$

    \end{proof}
    \item Para un polinomio
    \[
    P(x) = a_d x^d + \dots + a_0, \quad a_d \neq 0,
    \]
    tal que sus $d$ raíces son reales y distintas, se propone el siguiente método para aproximar todas sus raíces:
    \begin{itemize}
        \item Se comienza con un valor $x_0$ mayor que
        \[
        M = \max \left\{ 1, \sum_{i=0}^{d-1} \frac{|a_i|}{|a_d|} \right\}.
        \]
        \item Se genera a partir de $x_0$ la sucesión de Newton-Raphson, que, según el ítem anterior, converge a la raíz más grande de $P$, llamémosla $r_d$; obteniéndose de este modo un valor aproximado $\tilde{r_d}$.
        \item Se divide $P$ por $x - \tilde{r_d}$ y se desprecia el resto, dado que $r_d \approx \tilde{r_d}$. Se redefine ahora $P$ como el resultado de esta división y se comienza nuevamente desde el primer ítem, para hallar las otras raíces.
    \end{itemize}
    \item Aplicar este método para aproximar todas las raíces del polinomio
    \[
    P(x) = 2x^3 - 4x + 1.
    \]

    \begin{solution}
        Para implementar éste metodo en Matlab tendremos en cuenta varios aspectos importantes: si tenemos $P(x)= a_dx^d + \ldots + a_0$, su derivada es $P'(x)=da_dx^{d-1}+\ldots+a_2x+a_1$, con lo cual podemos definir dos funciones tales que dada una lista de coeficientes y una variable $x$, podemos evaluar $P(x)$ y $P'(x)$, además, sabemos que evaluar directamente es muy costoso computacionalmente, pues a medida que aumenta el grado del polinomio, hay que efectuar muchas operaciones tan solo para calcular $x^d$, por lo tanto escribimos $P(x)=x(x(\cdots(a_dx+a_{d-1})\cdots)+a_1)+a_0$ y hacemos lo mismo con $P'(x)$. Tambien consideramos que al dividir por $x-r_i$, podemos aplicar regla de Ruffini o división sintética, es decir, si tenemos $p(x)=a_nx^n + \cdots + a_0$ y dividimos entre $b(x)=x-r$ obtendremos un polinomio $q(x) = b_{n-1}x^{n-1} + \cdots + b_0 $ a partir de la siguiente regla de recurrencia:
        \begin{align*}
            b_{n-1} &= a_{n}\\
            b_{n-2} &= a_{n-1}+rb_{n-1}\\ 
            \vdots \\   
            b_{0} &= a_1 + rb_1
        \end{align*}
        Note que en este caso omitimos el residuo de la división, pues solo nos importa obtener los coeficientes. 

    Ya con todo esto, podemos aplicar el método de Newton-Rapshon al polinomio dado, de donde obtuvimos que sus raíces son $r_1 = -1.525687120865518$, $r_2 = 0.258652022504153$ y $r_3 = 1.267035098361366$. Al evaluar el polinomio en cada raíz obtenemos que $P(r_1) =  0.111022302462516 \times 10^{-15}$, $P(r_2)= -0.888178419700125 \times 10^{-15}$ y\\ $P(r_3)= -0.666133814775094 \times 10^{-15}$. Con esto podemos observar que el método de Newton-Raphson aproxima muy bien las raíces, pues aunque al hacer división sintética nos arriesgamos a perder un coeficiente, las aproximaciones resultantes son muy exactas, con una precisión de 15 cifras decimales.

    \end{solution}
\end{itemize}
