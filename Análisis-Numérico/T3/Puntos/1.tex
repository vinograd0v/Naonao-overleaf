%!TEX root = ../main.tex

\section*{Ejercicio 1}
Sea
\[
A = \begin{bmatrix} 0 & -20 & 14 \\ -3 & 27 & 4 \\ -4 & 11 & 2 \end{bmatrix}
\]
\begin{enumerate}
    \item[a)] Aplique las transformaciones de Householder, para calcular $A = QR$.
    \begin{solution}
        Veamos $A=(a_1|a_2|a_3)$ donde cada $a_i$ es el vecor columna correspondiente a la matriz del enunciado. Primero tenemos que
        $$\|a_1\|=\sqrt{0^2+(-3)^2+(-4)^2}=5,$$ 
        asi para construir nuestro reflector de HouseHolder tomamos
        $$v_1=a_1-\|a_1\|e_1=\begin{bmatrix}
            0\\-3\\-4
        \end{bmatrix}-\begin{bmatrix}
                5\\ 0\\0
            \end{bmatrix}=\begin{bmatrix}
                -5\\-3\\-4
            \end{bmatrix}.$$
            Luego tenemos que
            $$H_1=I_3-2\frac{v_1\cdot v_1^T}{v_1^T\cdot v_1}=\begin{bmatrix}
                1&0&0\\
                0&1&0\\
                0&0&1\\
            \end{bmatrix}-\frac{2}{50}\begin{bmatrix}
                25&15&20\\
                15&9&12\\
                20&12&16 
            \end{bmatrix}=\begin{bmatrix}
                0&-15/25&-20/25\\
                -15/25&16/25&-12/25\\
                -20/25&-12/25&9/25
            \end{bmatrix}.$$
            Luego haciendo la multiplicacion con $A$ tenemos
            $$H_1A=\begin{bmatrix}
                5&-25&-4\\
                0&24&-34/5\\
                0&7&-62/5
            \end{bmatrix}$$
            Ahora nos concentraremos en el sub-bloque $2\times 2$ de la matriz obtenida, es decir
            $$\tilde{A}=\begin{bmatrix}
                24&-34/5\\
                7&-62/5
            \end{bmatrix}$$
            Reallizando el proceso previo, si $\tilde{A}=(\tilde{a}_1|\tilde{a}_2),$ tenemos que
            $$\|\tilde{a}_1\|=\sqrt{24^2+7^2}=25,$$
            de esta manera
            $$v_2=\tilde{a}_1-\|\tilde{a}_1\|e_1=\begin{bmatrix}
                24\\
                7
            \end{bmatrix}-\begin{bmatrix}
                25\\
                0
            \end{bmatrix}=\begin{bmatrix}
                -1\\
                7
            \end{bmatrix}$$
            Si hacemos el reflector de HouseHolder para $\tilde{A}$ obtenemos
            $$\tilde{H}_2=I_2-2\frac{v_2\cdot v_2^T}{v_2^T\cdot v_2}=\begin{bmatrix}
                1&0\\
                0&1\\
            \end{bmatrix}-\frac{2}{50}\begin{bmatrix}
                1 & -7\\
                -7 & 49\\
            \end{bmatrix}=\begin{bmatrix}
                24/25&7/25\\
                7/25&-24/25
            \end{bmatrix}$$
            De esta manera el reflector para $A$ es
            $$H_2=\begin{bmatrix}
                1&0&0\\
                0&24/25&7/25\\
                0&7/25&-24/25
            \end{bmatrix}$$
            De esta manera tenemos que
            $$H_2H_1A=\begin{bmatrix}
                5& -25& -4\\
                0& 25& -10\\
                0& 0& 10
            \end{bmatrix}$$
            Note que esta matriz ya es triangular superior por lo que esa sera nuestra $R$, ahora como $H_1$ y $H_2$ son ortogonales y simetricos tenemos que:
            $$A=H_1^TH_2^TR=H_1H_2R,$$
            asi si realizamos la multiplicacion
            $$Q=H_1H_2=\begin{bmatrix}
                0&-20/25&15/25\\
                -15/25&12/25&16/25\\
                -20/25&-9/25&-12/25
            \end{bmatrix}.$$
            Concluyendo asi que la factorizacion $A=QR$ es
            $$\begin{bmatrix} 0 & -20 & 14 \\ -3 & 27 & 4 \\ -4 & 11 & 2 \end{bmatrix}=\begin{bmatrix}
                0&-20/25&15/25\\
                -15/25&12/25&16/25\\
                -20/25&-9/25&-12/25
            \end{bmatrix}\begin{bmatrix}
                5& -25& -4\\
                0& 25& -10\\
                0& 0& 10
            \end{bmatrix}$$

    \end{solution}
    \item[b)] Aplique el método de ortogonalización de Gram-Schmidt, para calcular $A = QR$.
    \begin{solution}
        Para este metodo tomamos nuevamente $A=(a_1|a_2|a_3)$, por medio de Gram-Schmidt conseguiremos vecotres ortonormales. Primero 
        $$q_1=\frac{a_1}{\|a_1\|}=\begin{bmatrix}
            0\\
            -3/5\\
            -4/5
        \end{bmatrix}$$
        Luego tomamos
        $$v_1=a_2-(q_1^Ta_2)q_1=\begin{bmatrix}
            -20\\
            27\\
            11
        \end{bmatrix}-(-25)\begin{bmatrix}
            0\\
            -3/5\\
            -4/5\\
        \end{bmatrix}=\begin{bmatrix}
            -20\\
            12\\
            -9\\
        \end{bmatrix}$$
        Asi nuestro vector ortonormal es
        $$q_2=\frac{v_1}{\|v_1\|}=\begin{bmatrix}
            -20/25\\
            12/25\\
            -9/25
        \end{bmatrix}$$
        De esta manera para nuestro ultimo vector tenemos 
        $$v_2=a_3-(q_1^Ta_3)q_1-(q_2^Ta_3)q_2=\begin{bmatrix}
            14\\
            4\\
            2
        \end{bmatrix}-(-4)\begin{bmatrix}
            0\\
            -3/5\\
            -4/5
        \end{bmatrix}-(-10)\begin{bmatrix}
            -20/25\\
            12/25\\
            -9/25
        \end{bmatrix}=\begin{bmatrix}
            6\\
             32/5\\
             -24/5\\
        \end{bmatrix}$$
        Asi nuestro ultimo vector es 
        $$q_3=\frac{v_2}{\|v_2\|}=\frac{1}{10}\begin{bmatrix}
            6\\
             32/5\\
             -24/5\\
        \end{bmatrix}=\begin{bmatrix}
            3/5\\
            16/25\\
            -12/25
        \end{bmatrix}$$
        De esta manera nuestra matriz ortogonal $Q$ es
        $$Q=(q_1|q_2|q_3)=\begin{bmatrix}
            0 &-20/25&3/5\\
            -3/5&12/25&16/25\\
            -4/5&-9/25&-12/25
        \end{bmatrix}=\begin{bmatrix}
                0&-20/25&15/25\\
                -15/25&12/25&16/25\\
                -20/25&-9/25&-12/25
            \end{bmatrix}$$
        Note que resulto ser la misma $Q$ hallada por medio de HouseHolder, luego natrualmente tenemos que
        $$R=Q^TA=\begin{bmatrix}
                5& -25& -4\\
                0& 25& -10\\
                0& 0& 10
            \end{bmatrix}$$
        Asi la factorizacion $A=QR$ resulta ser la misma que en el punto anterior, es decir
        $$\begin{bmatrix} 0 & -20 & 14 \\ -3 & 27 & 4 \\ -4 & 11 & 2 \end{bmatrix}=\begin{bmatrix}
                0&-20/25&15/25\\
                -15/25&12/25&16/25\\
                -20/25&-9/25&-12/25
            \end{bmatrix}\begin{bmatrix}
                5& -25& -4\\
                0& 25& -10\\
                0& 0& 10
            \end{bmatrix}$$
    \end{solution}
    \item[c)] Implemente en Matlab los métodos de ortogonalización de Gram-Schmidt y Householder, para calcular $A = QR$, compare los resultados numéricos con los encontrados en las partes (a) y (b).
\end{enumerate}